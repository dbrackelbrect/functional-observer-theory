\documentclass[12pt]{article}
\usepackage[a4paper, margin=1in]{geometry}
\usepackage{amsmath, amssymb, amsthm}
\usepackage{hyperref}
\usepackage{graphicx}
\usepackage{physics}
\usepackage{authblk}
\usepackage{titling}
\usepackage{mathrsfs}

\setlength{\droptitle}{-5em}   % Adjusts space above title

\title{Functional Observer Theory: A Delta-Preserving Interpretation of Quantum Mechanics}
\author{David Brackelbrect\\ \texttt{d.brackelbrect@protonmail.com}}
\date{June 11, 2025}

\begin{document}

\maketitle

\begin{abstract}
This paper introduces a novel interpretive framework in quantum mechanics wherein the observer neither collapses nor branches the quantum state. Instead, the observer executes a functional mapping operation that produces an interpretive derivative state from the original configuration, preserving its foundational identity, denoted $\delta$. This model aims to decouple time and measurement dependence from quantum evolution, replacing ontological transformation with functional recontextualization. Observation is defined as an intent-driven, delta-preserving interaction that elicits interpretive states without altering the base state itself. The theory posits that what we call time emerges not from the quantum state, but from the observer's invocation of function across interpretive layers.
\end{abstract}

\section{Introduction}
The standard interpretations of quantum mechanics --- Copenhagen, Many Worlds, QBism --- each attempt to reconcile the role of the observer in quantum state evolution. In this theory, we reject the assumption that observers induce change, collapse, or probabilistic updates in the ontological state. Instead, the observer enacts a mapping function $F$ that produces an interpretive state $\ket{\psi_f}$ derived from but not destructive to the original quantum state $\ket{\psi_0}$.

\section{Core Axioms of Functional Observer Theory}
\subsection*{Axiom 1: State Invariance}
\[ U \ket{\psi_0} = \ket{\psi_0}, \quad \forall U \in \mathcal{G} \]
The initial state is invariant under all operators in the group $\mathcal{G}$ defined by functional symmetry.

\subsection*{Axiom 2: Observation is Not Measurement}
\[ \mathcal{O} \nrightarrow \hat{M} \Rightarrow \mathcal{O} \rightarrow F : \ket{\psi_0} \mapsto \ket{\psi_f} \]
Observation does not collapse or measure. It invokes a function that produces an interpretive state.

\subsection*{Axiom 3: Functional Mapping}
\[ \ket{\psi_f} = F(\ket{\psi_0}, IO) \]
The functional state is derived from the base state and the observer's internal operation $IO$.

\subsection*{Axiom 4: Delta Preservation}
\[ \delta(\ket{\psi_f}, \ket{\psi_0}) = 0 \]
The interpretive state preserves the identity of the original state through a delta function.

\subsection*{Axiom 5: Temporal Emergence}
\[ T_{state} = \text{undefined}, \quad T_\mathcal{O} \in \mathbb{R} \]
Time is not intrinsic to the state but arises only through observer functional invocation.

\subsection*{Axiom 6: Retrofitted Choice}
\[ F(t_1) = F(t_0) \quad \text{where } t_1 > t_0 \]
Choices retroactively apply functionally; interpretive operations are atemporal in formulation.

\section{Interpretive Dynamics}
\subsection{Functional Emergence}
The observer applies $F$ in a way that instantiates one among many possible latent interpretive bubbles. Each invocation reconfigures the informational context, but not the quantum substrate.

\subsection{Latent Structures and Re-Evaluation}
Unchosen interpretive paths remain latent. They are not erased, but persist as structures influencing future observer function.

\subsection{Irreversibility and Causality}
Once a functional path is chosen, it forms part of an irreducible observer history. The base state is untouched; only the observer's interpretive chain evolves.

\subsection{Coherence vs. Divergence}
Multiple observers may functionally map different interpretive states from the same $\ket{\psi_0}$. This divergence is not a branching of reality, but a multiplicity of contextualized function outputs.

\section{Discussion and Implications}
This theory challenges collapse-driven and branching ontologies by postulating that quantum states are never altered, only recontextualized through functional logic. The role of the observer becomes active but non-invasive --- producing experience without altering underlying structure. It also presents a coherent framework for emergent time, entirely from observer-based recursion rather than spacetime metrics.

\section{Conclusion}
Functional Observer Theory posits that observation is a form of recontextualization rather than ontological interaction. By preserving $\delta$ across interpretive transitions, and generating time as an emergent consequence of intent-driven function, this framework invites a reconsideration of quantum realism, identity, and the operational status of the observer.

\section*{Acknowledgments}
The author thanks future discussion for its inevitable return.

\end{document}
