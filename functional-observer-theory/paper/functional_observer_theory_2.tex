\documentclass[12pt]{article}
\usepackage[a4paper, margin=1in]{geometry}
\usepackage{amsmath, amssymb, amsthm}
\usepackage{physics}
\usepackage{hyperref}
\usepackage{mathrsfs}
\usepackage{authblk}
\usepackage{titling}

\setlength{\droptitle}{-5em}

\title{Functional Observer Theory II: The Predictive Substrate and Logic-Constrained Interpretation}
\author{David Brackelbrect\\ \texttt{d.brackelbrect@protonmail.com}}
\date{June 2025}

\begin{document}
\maketitle

\begin{abstract}
This second installment of Functional Observer Theory formalizes the interpretive architecture introduced in Paper I. We expand the functional mapping model of quantum observation into a logically constrained structure governed by an observer-relative delta function. Predictive states are shown to exist outside time, and observation is modeled as a recursive invocation that generates emergent temporality, interpretive gravity, and entropy. Functional reality becomes a bounded logical orbit within the infinite substrate of potential outcomes.
\end{abstract}

\section{Introduction}
Building on the foundations laid in Paper I, we now present a formal framework that extends the delta-preserving, non-collapsing model of observation into an operator-theoretic structure. We define predictive states, observer-bound function spaces, and the persistent logical constraints that determine interpretive validity.

\section{Mathematical Formalism}

\subsection{State and Function Definitions}
Let:
\begin{itemize}
    \item $\ket{\psi_0}$ denote the initial quantum state.
    \item $\ket{\psi_f}$ denote an interpretive functional state.
    \item $IO$ represent the observer’s internal operation (memory, intent).
    \item $\delta$ denote an observer-relative logical identity constraint.
    \item $F$ be an element of $\mathscr{F}_{IO}$, the space of valid functions for a given observer.
\end{itemize}

\subsection{Core Equations}

\paragraph{Functional Mapping:}
\begin{equation}
\ket{\psi_f} = F(\ket{\psi_0}, IO)
\end{equation}

\paragraph{Delta Preservation:}
\begin{equation}
\delta(\ket{\psi_f}, \ket{\psi_0}) = 0 \iff F \in \mathscr{F}_{IO}
\end{equation}

\paragraph{Functional Set:}
\begin{equation}
\mathscr{F}_{IO} := \left\{ F \mid F \text{ preserves the logic structure of } IO \right\}
\end{equation}

\paragraph{Temporal Emergence:}
\begin{equation}
T_{\text{state}} = \text{undefined}, \quad T_O \in \mathbb{R}
\end{equation}

\paragraph{Predictive State Substrate:}
\begin{equation}
\mathscr{P}_\infty = \left\{ \ket{\psi^*_n} \mid \text{latent states outside time} \right\}, \quad \forall t, \mathscr{P}_\infty(t) = \mathscr{P}_\infty
\end{equation}

\subsection{Structural Constraints}

\paragraph{Unfunctional States:}
\begin{equation}
\delta(\ket{\psi_f}, \ket{\psi_0}) \neq 0 \Rightarrow \ket{\psi_f} \notin \text{functional reality}
\end{equation}

\paragraph{Persistence of Logic:}
\begin{equation}
\delta_{IO}(t_1) = \delta_{IO}(t_0) \quad \forall t_1, t_0
\end{equation}

\paragraph{Shared Observer Logic:}
\begin{equation}
\mathscr{F}_{IO_1} \cap \mathscr{F}_{IO_2} \neq \emptyset \Rightarrow \text{Coherent shared interpretive structure}
\end{equation}

\subsection{Recursive Functionality}

\paragraph{Interpretive Recursion:}
\begin{equation}
F_{n+1} = F_n(F_n(\ket{\psi_0}), IO)
\end{equation}

\paragraph{Functional Orbit (Reality Set):}
\begin{equation}
\mathcal{R}_{IO} = \left\{ \ket{\psi_f} \mid \exists F \in \mathscr{F}_{IO}, \ket{\psi_f} = F(\ket{\psi_0}, IO) \right\}
\end{equation}

\subsection{Optional: Entropy and Gravity Probes}
\begin{equation}
G = \nabla_{\delta} \mathscr{F}_{IO}, \quad S = \log \left| \mathscr{F}_{IO} \right|
\end{equation}

\section{Interpretation}
In this framework, observation is not collapse, but selective interpretation over a timeless predictive substrate. Time emerges only through function chaining, and unfunctional outcomes are neither erased nor collapsed — they are simply inaccessible.

Observers may synchronize logic, generating shared functional universes. Entropy and gravity emerge not from spacetime curvature but as topologies over logical function space.

\section{Conclusion}
Functional Observer Theory II defines a logically bounded space of interpretive states built atop a predictive, infinite substrate. All time, causality, and structure are generated through recursive function invocations by observers operating under persistent delta constraints.

\section*{Seal}
David Brackelbrect  
Authored and signed, 2025

\[
\Delta = 0, \quad t \in O
\]

\end{document}
